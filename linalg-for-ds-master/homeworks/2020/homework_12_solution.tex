\documentclass[11pt,nocut]{article}

\usepackage{../latex_style/packages}
\usepackage{../latex_style/notations}

\title{\vspace{-2.0cm}%
	Optimization and Computational Linear Algebra for Data Science\\
Homework 12: Gradient descent}
\date{}
\setcounter{section}{12}

\begin{document}
\maketitle
%%\noindent\textbf{Rules:}
\centerline{\pgfornament[width=13cm]{89}}
{\small
	\begin{itemize}
		\item Unless otherwise stated, all answers must be mathematically justified.
		\item Partial answers will be graded. 
		\item You can work in groups but each student must write his/her own solution based on his/her own understanding of the problem. Please list on your submission the students you work with for the homework (this will not affect your grade).
		\item Problems with a $(\star)$ are extra credit, they will not (directly) contribute to your score of this homework. However, for every $4$ extra credit questions successfully answered your lowest homework score get replaced by a perfect score.
		\item If you have any questions, feel free to contact me (\texttt{lm4271@nyu.edu}) or to stop at the office hours.
	\end{itemize}
}
\vspace{-0.4cm}
\centerline{\pgfornament[width=13cm]{89}}
\vspace{0.5cm}


%\vspace{1mm}

\begin{problem}[2 points] 
	\begin{enumerate}[label=\normalfont(\textbf{\alph*})]
		\item $f$ has a local maximum at $(1.2,1.3)$ and a global maximum at $(-0.5,-0.7)$. $f$ has a local minimum at $(-0.9,0.7)$ and a global minimum at $(0.9,-0.9)$ and a saddle-point at $(0.9,0)$.
		\item When initialized at $A$, gradient descent is likely to converge to the local minimum at $(-0.9,0.7)$. When initialized at $B$, gradient descent is likely to converge to the global minimum at $(0.9,-0.9)$.
	\end{enumerate}
\end{problem}

\vspace{5mm}

\begin{problem}[5 points]\label{p:grad}
	$$
	f(x) = \frac{1}{2} x^{\sT} M x - \langle x,b \rangle + c
	$$
	\begin{enumerate}[label=\normalfont(\textbf{\alph*})]
		\item Let $x \in \R^d$. $f$ is twice differentiable and
			$$
			H_f(x) = M.
			$$
			By definition of $\mu$ and $L$, the eigenvalues of $M$ are all above $\mu$ and all smaller than $L$: $f$ is therefore $\mu$-strongly convex and $L$-smooth. $f$ is therefore convex. Hence 
			\begin{align*}
				x \text{ is a global minimizer of } f
				& \ \ \Longleftrightarrow \ \ \nabla f(x) =0 \\
				& \ \ \Longleftrightarrow \ \ Mx -b =0 \\
				& \ \ \Longleftrightarrow \ \ x = M^{-1} b.
			\end{align*}
			$x^* = M^{-1} b$ is therefore the unique global minimizer of $f$.

		\item $\nabla f(x) = Mx -b$ hence
			\begin{align*}
				x_{t+1} - x^* 
				&= x_t - x^* - \beta (Mx_t -b)
				= x_t - x^* - \beta M (x_t - M^{-1} b)
				\\
				&= x_t - x^* - \beta (x_t -x^*)
				\\
				&= (\Id - \beta M)(x_t - x^*).
			\end{align*}
		\item Let $B = \Id - \beta M$. $B$ is symmetric and his eigenvalues are:
			$$
			1 - \lambda_1 / L, \dots, 1 - \lambda_d/L
			$$
			which are all between $0$ and $1- \mu/L$. The largest eigenvalue of $B^2$ is therefore $(1-\mu/L)^2$. Since the singular values of $B$ are the square root of the eigenvalues of $B^{\sT} B= B^2$ because $B$ is symmetric, we get that the largest singular value of $B$ is $1-\mu/L$.

			We know that the spectral norm of a matrix is equal to its largest singular value: 
			$\|B\|_{\rm Sp} = 1 - \mu /L$. Hence
			$$
			\|x_{t+1} - x^* \| = 
			\| B (x_t-x^*)\| \leq \|B\|_{\rm Sp} \|x_t-x^*\| =
			\Big(1- \frac{\mu}{L}\Big) \|x_t - x^*\|,
			$$
			from which the result follows.
		\item Since $w_{t+1} = (\Id - L^{-1} M) w_t$, we have for $i \in \{1, \dots, d\}$
			$$
			\alpha_i(t+1) 
			= v_i^{\sT} (\Id - L^{-1} M) w_t
			= (v_i^{\sT} - L^{-1} v_i^{\sT} M) w_t.
			$$
			Now, we use the fact that $Mv_i = \lambda_i v_i$ to get $v_i^{\sT} M = \lambda_i v_i^{\sT}$:
			$$
			\alpha_i(t+1)  = (1- \lambda_i/L) \alpha_i(t).
			$$
			This gives
			$$
			\alpha_i(t) = (1-\lambda_i/L)^t \alpha_i(0).
			$$
		\item Let $i\in \{1, \dots, d\}$. $|\alpha_i(t)| = |\langle v_i, x_t-x^* \rangle|$ is equal to the norm of the orthogonal projection of $x_t - x^*$ onto $\Span(v_i)$, that is corresponds to <<the distance between $x_t$ and $x^*$ in the direction of $v_i$>>.
			\\

			From the previous we see that at each iteration of gradient descent, this <<distance>> is multiplied by a factor $1-\lambda_i/L$, where $\lamdba_i \in [\mu,L]$. Hence, gradient descent converges faster <<in the direction of $v_i$>> if $\lambda_i$ is large (close to $L$).

		\item $(\alpha_1(t) , \dots, \alpha_d(t))$ are the coordinates of $w_t = x_t - x^*$ in the orthonormal basis $(v_1, \dots,v_d)$. Therefore
			$$
			\|x_t - x^* \| = \sqrt{\sum_{i=1}^d \alpha_i(t)^2} = \sqrt{\sum_{i=1}^d \Big(1-\frac{\lambda_i}{L}\Big)^{\!2t} \big\langle v_i, x_0-x^* \big\rangle^2}.
			$$
	\end{enumerate}
\end{problem}



\vspace{1cm}
\centerline{\pgfornament[width=7cm]{87}}

%\bibliographystyle{plain}
%\bibliography{./references.bib}
\end{document}

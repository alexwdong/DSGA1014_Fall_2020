\documentclass[11pt,nocut]{article}

\usepackage{../latex_style/packages}
\usepackage{../latex_style/notations}

\title{\vspace{-2.0cm}%
	Optimization and Computational Linear Algebra for Data Science\\
Homework 11: Optimality conditions}
\date{}
\author{Due on December 6, 2019}
\setcounter{section}{11}

\begin{document}
\maketitle
%\noindent\textbf{Rules:}
\centerline{\pgfornament[width=13cm]{89}}
{\small
	\begin{itemize}
		\item Unless otherwise stated, all answers must be mathematically justified.
		\item Partial answers will be graded. 
		\item You can work in groups but each student must write his/her own solution based on his/her own understanding of the problem. Please list on your submission the students you work with for the homework (this will not affect your grade).
		\item Problems with a $(\star)$ are extra credit, they will not (directly) contribute to your score of this homework. However, for every $4$ extra credit questions successfully answered your lowest homework score get replaced by a perfect score.
		\item If you have any questions, feel free to contact me (\texttt{lm4271@nyu.edu}) or to stop at the office hours.
	\end{itemize}
}
\vspace{-0.4cm}
\centerline{\pgfornament[width=13cm]{89}}
\vspace{0.5cm}


\vspace{1mm}

\begin{problem}[2 points]
	Let $f,g: \R^3 \to \R$ be the functions defined by
	$$
	f(x,y,z) = 2x^2 + y^2 + \frac{1}{2} z^2 + 4x - 6y - z + 1
	\quad \text{and} \quad
	g(x,y,z) =-xyz + x+y+z .
	$$
	Compute the critical points of $f$ and $g$ and determine if they are global/local maximizer, local minimizer or saddle point.
\end{problem}

\vspace{5mm}

\begin{problem}[3 points]
	We consider the following constrained optimization problem:
	\begin{equation}\label{eq:p}
	\text{minimize} \quad x-y+z \quad \text{subject to} \quad x^2 + y^2 + z^2=1 \quad \text{and} \quad x+y+z=1.
	\end{equation}
	We admit that this minimization problem has (at least) one solution (this comes from the fact that a continuous function on a compact set attains its minimum).
	\\

		Using Lagrange multipliers, show that \eqref{eq:p} has a unique solution and compute its coordinates.
\end{problem}

\vspace{5mm}

\begin{problem}[2 points]
	Let $u \in \R^n$ be a vector such that for all $i \neq j$, $|u_i| \neq |u_j|$. We consider the constrained optimization problem
	$$
	\text{maximize} \quad \langle u ,x \rangle \quad \text{subject to} \quad \|x\|_1 \leq 1.
	$$
	\begin{enumerate}[label=\normalfont(\textbf{\alph*})]
		\item Show that this problem has a unique solution $x^*$ and give the expression of $x^*$ in terms of $u$ (Lagrange multipliers are not needed here).
		\item Give a graphical interpretation.
	\end{enumerate}
\end{problem}

\newpage

\begin{problem}[3 points]
	\emph{\textbf{We will prove the spectral theorem in this problem: you are therefore not allowed to use the spectral theorem and its consequences to solve this exercise.}}

	Let $A$ be an $n \times n$ symmetric matrix. We consider the following optimization problem
	\begin{equation}\label{eq:eig1}
		\text{maximize} \quad x^{\sT} A x \quad \text{subject to} \quad \|x\| = 1.
	\end{equation}
	This optimization problem admits a solution (this comes from the fact that a continuous function on a compact set achieved its maximum) that we denote by $v_1$.
	\begin{enumerate}[label=\normalfont(\textbf{\alph*})]
		\item Using Lagrange multipliers, show that $v_1$ is an eigenvector of $A$.
		\item We now consider the optimization problem
	\begin{equation}\label{eq:eig2}
		\text{maximize} \quad x^{\sT} A x \quad \text{subject to} \quad \|x\| = 1
		\quad \text{and} \quad \langle x,v_1 \rangle = 0.
	\end{equation}
	For the same reason as above, this problem admits a solution that we denote by $v_2$. Show that $v_2$ is an eigenvector of $A$ that is orthogonal to $v_1$.
		\item We now consider the optimization problem
	\begin{equation}\label{eq:eig3}
		\text{maximize} \quad x^{\sT} A x \quad \text{subject to} \quad \|x\| = 1
		\quad \text{and} \quad \langle x,v_1 \rangle = 0
		\quad \text{and} \quad \langle x,v_2 \rangle = 0.
	\end{equation}
	Again, this problem admits a solution that we denote by $v_3$. Show that $v_3$ is an eigenvector of $A$ that is orthogonal to $v_1$ and $v_2$.
	\end{enumerate}
	\emph{
		\textbf{Conclusion}: by repeating this procedure, we obtain an orthonormal family $v_1, \dots, v_n$ of eigenvectors of $A$. This proves the spectral theorem (without using any linear algebra result!).
	}
\end{problem}


\vspace{5mm}

\begin{problem}[$\star$]
	We consider here a simple portfolio optimization problem.
	Assume that we can invest in $n$ financial assets. Each asset $i$ has a return of $X_i$ ($X_i$ is a random variable) meaning that investing $w$\$ in the asset $i$ will generate a return of $w \times X_i$ \$. We introduce
	\begin{align*}
		r_i &= \E[X_i], \\
		\Sigma &= \Cov(X,X) = \E\big[(X-r) (X-r)^{\sT}\big]
	\end{align*}
	that are respectively the average return and the covariance matrix of the returns. 
	We assume that the covariance matrix $\Sigma$ is invertible. We represent a portfolio (an investment strategy) by a vector $w = (w_1, \dots, w_n) \in \R^n$, where each coordinate $w_i$ represents the amount invested on the asset $i$. 
	We allow $w_i$ to be negative, which means that it is possible to ``short'' a security.
	The expected return of the portfolio is then
	$$
	R(w) = \E \left[\sum_{i=1}^n w_i X_i\right] = \sum_{i=1}^n w_i r_i
	$$
	and its variance is
	\begin{align*}
	V(w) 
	&= \Var \left(\sum_{i=1}^n w_i X_i\right) 
	= \E \left[\left(\sum_{i=1}^n w_i (X_i -r_i)\right)^2\right]
	= \E \left[\sum_{i=1}^n\sum_{j=1}^n w_iw_j (X_i -r_i)(X_j -r_j)\right]
	\\
	&= w^{\sT} \Sigma w.
	\end{align*}
	We want here to invest a total amount of $m$\$ on the assets. 
	Given an targeted expected return $\mu$, we would like to find a portfolio $w$ such that $R(w)=\mu$ and $\sum_{i=1}^n w_i = m$ for which the variance $V(w)$ is minimal.
	That is, we aim at solving the following problem:
	\begin{equation}\label{mar}
		\text{minimize} \quad
		V(w) \quad \text{subject to} \quad
		R(w) = \mu \quad \text{and} \quad \sum_{i=1}^n w_i = m.
	\end{equation}
	Let $\1$ denotes the all-ones vector of dimension $n$. We assume that $r \not\in \Span(\1)$.
	%We also assume that $m \min_i r_i < \mu < m \max_i r_i$, so that there exists a feasible $w$ to the problem \eqref{mar}.
	Show that the matrix
	$$
	M = 
	\begin{pmatrix}
		\1^{\sT} \Sigma^{-1} \1 & \1^{\sT} \Sigma^{-1} r \\
		\1^{\sT} \Sigma^{-1} r & r^{\sT} \Sigma^{-1} r
	\end{pmatrix}
	$$
	is invertible and that the unique solution to \eqref{mar} is
	$$
	w^* = \Sigma^{-1} 
	\begin{pmatrix}
		| & | \\
		\1 & r \\
		| & |
	\end{pmatrix}
	M^{-1} 
	\begin{pmatrix}
		m \\
		\mu
	\end{pmatrix}.
	$$
\end{problem}



\vspace{1cm}
\centerline{\pgfornament[width=7cm]{87}}

%\bibliographystyle{plain}
%\bibliography{./references.bib}
\end{document}

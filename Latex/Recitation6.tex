\documentclass{beamer}
\usepackage{amsthm}
\usepackage[utf8]{inputenc}
\usetheme{Madrid}
\usepackage{../latex_style/packages_old}
\usepackage{../latex_style/notations_old}
\usepackage{outlines}
\usepackage{enumitem}
\renewenvironment{itemize}
\usefonttheme{serif}
\def\labelenumi{\theenumi}
\usefonttheme{serif}

%\setbeamertemplate{itemize items}[default]
%\setbeamertemplate{enumerate items}[default]

% define smaller font command
\newcommand*{\horzbar}{\rule[.5ex]{2.5ex}{0.5pt}}
\newcommand\fonteight{\fontsize{8}{9.6}\selectfont}
\newcommand\fontten{\fontsize{10}{1.2}\selectfont}

%define enumerate with periods
\renewenvironment{enumerate}%
{\begin{list}{\arabic{enumi}.}%  \langle ------ dot here
      {\setlength{\leftmargin}{2.5em}%
       \setlength{\itemsep}{-\parsep}%
       \setlength{\topsep}{-\parskip}%%
       \usecounter{enumi}}%
 }{\end{list}}
%define itemize with arrows
\renewenvironment{itemize}%
{\begin{list}{$\blacktriangleright$}%  \langle ------ dot here
      {\setlength{\leftmargin}{2.5em}%
       \setlength{\itemsep}{-\parsep}%
       \setlength{\topsep}{-\parskip}%%
       \usecounter{enumi}}%
 }{\end{list}}

%Information to be included in the title page:
\title{Recitation 6}
\author{Alex Dong}
\institute{CDS, NYU}
\date{Fall 2020}


\makeatletter
\setbeamertemplate{navigation symbols}{}
\setbeamertemplate{footline}{
  \leavevmode%
  \hbox{%
  \begin{beamercolorbox}[wd=.4\paperwidth,ht=2.25ex,dp=1ex,center]{author in head/foot}%
    \usebeamerfont{author in head/foot}\insertshortauthor\expandafter\ifblank\expandafter{\beamer@shortinstitute}{}{~~(\insertshortinstitute)}
  \end{beamercolorbox}%
  \begin{beamercolorbox}[wd=.3\paperwidth,ht=2.25ex,dp=1ex,center]{title in head/foot}%
    \usebeamerfont{title in head/foot}\insertshorttitle
  \end{beamercolorbox}%
  \begin{beamercolorbox}[wd=.3\paperwidth,ht=2.25ex,dp=1ex,right]{date in head/foot}%
    \usebeamerfont{date in head/foot}\insertshortdate{}\hspace*{2em}
    \insertframenumber{} / \inserttotalframenumber\hspace*{2ex} 
  \end{beamercolorbox}}%
  \vskip0pt%
}
\setbeamertemplate{navigation symbols}{}
\makeatother

\begin{document}
%1
\frame{\titlepage} 
%2

\begin{frame}
\frametitle{Etymology}

\begin{itemize}
\item \textit{eigen}values and \textit{eigen}vectors
\item What does \textit{eigen} mean anyway?
\item German word for...
\begin{enumerate}
\item own
\item innate
\item peculiar
\item \textbf{intrinsic}
\end{enumerate}
\item A square matrix `owns' certain vectors... or there are certain vectors that are intrinsic to a matrix.
\end{itemize}
\end{frame}

\begin{frame}
\frametitle{Importance of Eigenvalues and Eigenvectors}
\center{!!! \textit{\textbf{SERIOUSLY IMPORTANT}} !!!}\\
\bigskip
\begin{itemize}
\item Eigen-val/vec will show up \textit{continuously} throughout this course
\item Connections to...
\begin{itemize}
\item Projections and Orthogonal Projections (Lec 4)
\item Markov Chains (Lec 6)
\item Spectral Theorem (HW 6, Lec 7)
\item SVD (Lec 7)
\item Spectral Clustering (!!??) (Lec 8)
\item Positive definite and positive semi-definite matrices (Lec 10,11)
\end{itemize}
\item Many other applications not covered in this course
\item Literally cannot stress this enough

\end{itemize}
\end{frame}


\begin{frame}
\frametitle{$Av=\lambda v$. So what's the big deal?}
\begin{itemize}
\item Square matrices are important enough to get their own name - \textit{operators}.
\item Sometimes a matrix A `prefers' certain directions
\item These directions are useful 'anchors' to understanding what a matrix does.
\item (&)
\end{itemize}



\end{frame}
\end{document}
\documentclass{beamer}
\usepackage{amsthm}
\usepackage[utf8]{inputenc}
\usetheme{Madrid}
\usepackage{../latex_style/packages_old}
\usepackage{../latex_style/notations_old}
\usepackage{outlines}
\usepackage{enumitem}
\renewenvironment{itemize}
\usefonttheme{serif}
\def\labelenumi{\theenumi}
\usefonttheme{serif}
\usepackage{comment}

%\setbeamertemplate{itemize items}[default]
%\setbeamertemplate{enumerate items}[default]

% define smaller font command
\newcommand*{\horzbar}{\rule[.5ex]{2.5ex}{0.5pt}}
\newcommand\fonteight{\fontsize{8}{9.6}\selectfont}
\newcommand\fontten{\fontsize{10}{1.2}\selectfont}

%define enumerate with periods
\renewenvironment{enumerate}%
{\begin{list}{\arabic{enumi}.}%  \langle ------ dot here
      {\setlength{\leftmargin}{2.5em}%
       \setlength{\itemsep}{-\parsep}%
       \setlength{\topsep}{-\parskip}%%
       \usecounter{enumi}}%
 }{\end{list}}
%define itemize with arrows
\renewenvironment{itemize}%
{\begin{list}{$\blacktriangleright$}%  \langle ------ dot here
      {\setlength{\leftmargin}{2.5em}%
       \setlength{\itemsep}{-\parsep}%
       \setlength{\topsep}{-\parskip}%%
       \usecounter{enumi}}%
 }{\end{list}}

%Information to be included in the title page:
\title{Recitation 6}
\author{Alex Dong}
\institute{CDS, NYU}
\date{Fall 2020}


\makeatletter
\setbeamertemplate{navigation symbols}{}
\setbeamertemplate{footline}{
  \leavevmode%
  \hbox{%
  \begin{beamercolorbox}[wd=.4\paperwidth,ht=2.25ex,dp=1ex,center]{author in head/foot}%
    \usebeamerfont{author in head/foot}\insertshortauthor\expandafter\ifblank\expandafter{\beamer@shortinstitute}{}{~~(\insertshortinstitute)}
  \end{beamercolorbox}%
  \begin{beamercolorbox}[wd=.3\paperwidth,ht=2.25ex,dp=1ex,center]{title in head/foot}%
    \usebeamerfont{title in head/foot}\insertshorttitle
  \end{beamercolorbox}%
  \begin{beamercolorbox}[wd=.3\paperwidth,ht=2.25ex,dp=1ex,right]{date in head/foot}%
    \usebeamerfont{date in head/foot}\insertshortdate{}\hspace*{2em}
    \insertframenumber{} / \inserttotalframenumber\hspace*{2ex} 
  \end{beamercolorbox}}%
  \vskip0pt%
}
\setbeamertemplate{navigation symbols}{}
\makeatother

\begin{document}
%1
\frame{\titlepage} 
%2

\begin{frame}
\frametitle{Stochastic Processes Rabbit Hole}
\begin{itemize}
\item Markov Chains are a topic in \textit{stochastic processes}
\item Stochastic Processes (\&)
\begin{itemize}
\item Finite Markov Chains (Discrete time, Discrete space)
\item Infinite Markov Chains (Discrete time, inf. discrete space)
\item Poisson Process (Continuous time, Discrete space)
\item Brownian Motion (Continuous time, continuous space)
\end{itemize}
\item Main Assumption: Markov Property (Memoryless)
\item Lots of Linear Algebra!
\end{itemize}
\end{frame}



\begin{frame}
\frametitle{Questions: Stochastic Matrices}
Let $A,B\in \R^{n\times n}$ be  stochastic matrices.
True or False for 1,2,3.
\begin{enumerate}
\item $A$ is always invertible
\item The eigenvector corresponding to the largest eigenvalue of $A$ is unique
\item $A$ cannot have zero has its eigenvalue
\item Prove that $AB$ is a stochastic matrix.
\end{enumerate}

Hint for 4. Is there a way express the "sum of each column is 1" property as a matrix multiplication?
\end{frame}

\begin{frame}
\frametitle{Solutions 1: Stochastic Matrices}
Let $A$ be a stochastic matrix.
True or False for 1,2,3.
\begin{solution}
\begin{enumerate}
\item $A$ is always invertible\\
False 
$\begin{bmatrix}
1 & 1 \\
0 & 0 \\
\end{bmatrix}$
\item The eigenvector corresponding to the largest eigenvalue of $A$ is unique.
False 
$\begin{bmatrix}
1 & 0 \\
0 & 1 \\
\end{bmatrix}$
\item $A$ cannot have zero has its eigenvalue.
False:
$\begin{bmatrix}
1 & 1 \\
0 & 0 \\
\end{bmatrix}$
\end{enumerate}
\end{solution}
\end{frame}

\begin{frame}
\frametitle{Solutions 2: Stochastic Matrices}
\begin{enumerate}
\item[4.] Prove that the product of two stochastic matrices is a stochastic matrix.
\begin{solution}
Let $A,B$ be stochastic matrices in $\R^n$. \\
Each entry is non-negative: \\
$AB_{i,j} = \sum_{k=1}^n A_{i,k}B_{k,j}$\\
This summation is a sum of non-negative products, hence it is also non-negative.

Sum of each columns is 1: \\

Note that the property of each column summing to 1 can be seen as: \\
A matrix $A$ is stochastic when \\
$\begin{bmatrix}
1& \hdots &1 \end{bmatrix} A = \begin{bmatrix}
1& \hdots &1 \end{bmatrix}$. 
\\
Then \\
$  
\begin{bmatrix}
1& \hdots &1 \end{bmatrix} AB = \begin{bmatrix}
1& \hdots &1 \end{bmatrix}A = \begin{bmatrix}
1& \hdots &1 \end{bmatrix}$. \\
So $AB$ is stochastic.
\end{solution}
\end{enumerate}
\end{frame}

\begin{frame}
\frametitle{Change of Basis}
\begin{itemize}
\item Sometimes, a  matrix $A$ `prefers' certain directions (\textit{eigenvectors})
\item (!!!) These directions act as \textit{anchors} for understanding the action of a matrix.
\item These are directions that we will \textit{orient} or \textit{change our basis} to.
\item This is related to $P^{-1}DP$, or diagonalization (Lec 7). 
\begin{itemize}
\item Defined as $P^{-1}DP$ or $PDP^{-1}$, depending on which text/notes you reference. 
\end{itemize}
\end{itemize}
\end{frame}


\begin{frame}
\frametitle{Question: Change of Basis}
Let $A$ have eigenvectors $v_1,...,v_n$ with corresponding eigenvalues $\lambda_1,...,\lambda_n$.
Let $\mathcal{B} = \{v_1,...,v_n\}$ be a basis of $\R^n$. Let $x = \sum_{i=1}^n \alpha_iv_i$
\begin{enumerate}
\item Let $P$ be a linear transformation that maps (canoncial basis vectors) $e_i$ to $v_i$, for all $i\in{1,...,n}$. Write the matrix $P$.
\item What is $PDP^{-1} x?$
\item Let $k\in \N$. What is $(PDP^{-1})^kx$?
\item If $A = PDP^{-1}$, give an interpretation for the action of $A$.
\end{enumerate}
\end{frame}

\begin{frame}
\frametitle{Solutions 1: Change of Basis}
Let $A$ have eigenvectors $v_1,...,v_n$ with corresponding eigenvalues $\lambda_1,...,\lambda_n$.
Let $\mathcal{B} = \{v_1,...,v_n\}$ be a basis of $\R^n$. Let $x = \sum_{i=1}^n \alpha_iv_i$
\begin{solution}
\begin{enumerate}

\item Let $P$ be a linear transformation that maps (canoncial basis vectors) $e_i$ to $v_i$, for all $i\in{1,...,n}$. Write the matrix $P$.

$P = 
\begin{bmatrix}
\vert & & \vert \\
v_1 & \hdots & v_n \\
\vert & & \vert \\
\end{bmatrix}
$
\item Let $D = Diag(\lambda_1,...,\lambda_n)$. What is $PDP^{-1} x?$\\
\qquad $P^{-1}x = P^{-1}\sum_{i=1}^n \alpha_i v_i =  \sum_{i=1}^n \alpha_i e_i$ \\
\qquad $DP^{-1}x = D(\sum_{i=1}^n \alpha_i e_i) =  \sum_{i=1}^n \lambda_i\alpha_i e_i$\\
\qquad $PDP^{-1}x = P(\sum_{i=1}^n \lambda_i\alpha_i e_i) =  \sum_{i=1}^n \lambda_i\alpha_i v_i$\\

\end{enumerate}

\end{solution}
\end{frame}


\begin{frame}
\frametitle{Solutions 2: Change of Basis}
Let $A$ have eigenvectors $v_1,...,v_n$ with corresponding eigenvalues $\lambda_1,...,\lambda_n$.
Let $\mathcal{B} = \{v_1,...,v_n\}$ be a basis of $\R^n$. Let $x = \sum_{i=1}^n \alpha_iv_i$
\begin{solution}
\begin{enumerate}

\item[3.] Let $k\in \N$ What is $(PDP^{-1})^kx$?\\
\qquad $(PDP^{-1})^2 = PDP^{-1}PDP^{-1} = PD^2P^{-1}$ \\
Likewise,\\
\qquad $(PDP^{-1})^k = PD^kP^{-1}$ \\

\item[4.] If $A = PDP^{-1}$, give an interpretation for the action of $A$ on a vector $x$.\\
First view $x$ as a vector in the coordinates of basis $v_1,...,v_n$. \\ 
$P^{-1}$ transforms these coordinates into the standard basis. \\
$D$ stretches the new coordinate $e_i$ by the eigenvalue $\lambda_i$. \\
$P$ transforms these coordinates back into the basis of $v_1,...,v_n$ \\

\end{enumerate}
\end{solution}
\end{frame}

\begin{frame}
\frametitle{Spectral Theorem}

\begin{theorem}[Spectral Theorem (!!!)]\label{th:spectral}
	Let $A \in \R^{n \times n}$ be a \textbf{symmetric} matrix. Then there is a orthonormal basis of $\R^n$ composed of eigenvectors of $A$.
\end{theorem}

\begin{itemize}
\item One of most important theorems in Linear Algebra
\item Symmetric matrices induce an orthonormal basis of eigenvectors
\item Makes analyzing the behavior of a symmetric matrix very easy
\item Many examples of 'natural' symmetric matrices
\begin{itemize}
\item Covariance Matrix
\end{itemize}
\item Proved in Homework 5
\end{itemize}
\end{frame}



\begin{frame}
\frametitle{Questions: Spectral Theorem}

\begin{enumerate}
\item Let $A\in\R^{n\times n}$ be a symmetric matrix.
 Give a vector $v$ with $\|v\|=1$ such that $\|Av\|$ is maximized.\\
 \medskip
\item Let $A\in\R^{n\times n}$ be a symmetric matrix, with eigenvalues $\lambda_1,...,\lambda_n$, and orthonormal family of associated eigenvectors $u_1,...,u_n$. Give an orthonormal basis of $Ker(A)$ and $Im(A)$ in terms of the $u_i$'s.
\end{enumerate}
\end{frame}

\begin{frame}


\frametitle{Solutions 1: Spectral Theorem}

\begin{enumerate}
\item Let $A\in\R^{n\times n}$ be a symmetric matrix. Give a vector $v$ with $\|v\|=1$ such that $\|Av\|$ is maximized.
\begin{solution}

Let $A = U \Lambda U^T$, where $U$ is orthogonal with columns $u_1,...,u_n$, and $\Lambda$ is diagonal with entries $\lambda_1,...,\lambda_n$. \\
Let $v = \sum_{i=1}^n \alpha_i u_i$, where $\sum_{i=1}^n \alpha_i^2 = 1$\\  
Then,\\
\qquad $U^Tv = \sum_{i=1}^n \alpha_i e_i$.\\
\qquad $\Lambda U^Tv = \sum_{i=1}^n \lambda_i \alpha_i e_i$ \\
\qquad $U \Lambda U^Tv = \sum_{i=1}^n \lambda_i \alpha_i u_i$ \\
\qquad $\|U \Lambda U^Tv\| = \|\sum_{i=1}^n \lambda_i \alpha_i u_i\|$ \\
\qquad $\|U \Lambda U^Tv\| = \sum_{i=1}^n \lambda_i^2 \alpha_i^2 \|u_i\|$ \qquad by orthonormality \\
\qquad $\|U \Lambda U^Tv\| = \sum_{i=1}^n \lambda_i^2 \alpha_i^2$\\
Maximize this quantity by setting $\alpha_j = 1$,  where $j$ is the index with the largest magnitude eigenvalue.
 
\end{solution}
\end{enumerate}
\end{frame}


\begin{frame}
\frametitle{Solutions 2 : Spectral Theorem}

\begin{enumerate}
\item[2.] Let $A\in\R^{n\times n}$ be a symmetric matrix, with eigenvalues $\lambda_1,...,\lambda_n$, and orthonormal family of associated eigenvectors $u_1,...,u_n$. Give an orthonormal basis of $Ker(A)$ and $Im(A)$ in terms of the $u_i$'s.
\begin{solution}
Since $A$ is symmetric, then $A$ induces an orthonormal basis of eigenvectors $u_1,...,u_n$ with eigenvalues $\lambda_1,...,\lambda_n$, and $A = U\Lambda U^T$.\\
Let $v = \sum_{i=1}^n \alpha_i u_i$ \\  
Then by the previous question,\\
\qquad $U \Lambda U^Tv = \sum_{i=1}^n \lambda_i \alpha_i u_i$ \\
Now, consider all $j$ s.t $\lambda_j = 0$. \\
$Av = \sum_{i=1}^n \lambda_i \langle v,u_i\rangle = 0$ when $v \in Span(\{u_j| \lambda_j=0,\text{ for } j\in 1,...,n\})$.\\
So $Ker(A) = Span(\{u_j| \lambda_j=0,\text{ for }j\in 1,...,n\})$.\\
Likewise,\\ 
$Im(A) = Span(\{u_k | \lambda_k \neq 0,\text{ for } k\in 1,...,n\} )$

\end{solution}

\end{enumerate}
\end{frame}


\begin{comment}
\begin{frame}
\frametitle{Solution 1: Change of Basis 1}

Let $\mathcal{B} = \{v_1,...,v_n\}$ be a basis of $\R^n$.
\begin{solution}
\begin{enumerate}
\item  \\ 
$P = 
\begin{bmatrix}
\vert & & \vert \\
v_1 & \hdots & v_n \\
\vert & & \vert \\
\end{bmatrix}
$\\
\smallskip
\item Is $P$ invertible?\\ 
Since $v_1,...,v_n$ are linearly indepedent, $rank(P) = n$ so $P$ is invertible.\\
\smallskip 

\end{enumerate}
\end{solution}
\end{frame}

\begin{frame}
\frametitle{Solution 2: Change of Basis 1}

Let $\mathcal{B} = \{v_1,...,v_n\}$ be a basis of $R^n$.
\begin{solution}
\begin{enumerate}

\item[3.] Let $x\in \R^n$ s.t $x = \sum_{i=1}^n \alpha_i e_i$. Write an expression for $Px$ in terms of $v_i$.\\
$Px = P(\sum_{i=1}^n \alpha_i e_i) = \sum_{i=1}^n \alpha_i P(e_i) = \sum_{i=1}^n \alpha_i v_i$ \\
So,
$x = 
\begin{bmatrix}
\alpha_1 \\
\vdots \\
\alpha_n \\
\end{bmatrix}
$ and 
$Px = 
\begin{bmatrix}
\vert \\
\sum_{i=1}^n \alpha_i v_i \\
\vert \\
\end{bmatrix}$
 or 
$Px = \begin{bmatrix}
\sum_{i=1}^n \alpha_i v_i[1] \\
\vdots \\
\sum_{i=1}^n \alpha_i v_i[n] \\
\end{bmatrix}
$
\end{enumerate}
\end{solution}
\end{frame}

\begin{frame}
\frametitle{Solution 2: Change of Basis 1}

Let $\mathcal{B} = \{v_1,...,v_n\}$ be a basis of $R^n$.
\begin{solution}
\begin{enumerate}

\item[4.] Describe what the action of $D$
If we think of a vector $x$ as a sum of $/alpha_ie_i$, P transforms each $e_i$ into $v_i$

\end{enumerate}
\end{solution}
\end{frame}


\begin{frame}
\frametitle{Questions: Change of Basis 2}

Let $\mathcal{B} = \{v_1,...,v_n\}$ be a basis of $\R^n$, and $P = v_1,...,v_n$. and $D = Diag(\lambda_1,...,\lambda_n)$
\begin{enumerate}
\item Conceptually, describe the action of $D$.
\item Conceptually, describe the action of $P^{-1}DP$
\end{enumerate}
\end{frame}

\begin{frame}
\frametitle{Solutions: Change of Basis 2}

Let $\mathcal{B} = \{v_1,...,v_n\}$ be a basis of $\R^n$, and $P = v_1,...,v_n$. and $D = Diag(\lambda_1,...,\lambda_n)$
\begin{enumerate}
\begin{solution}
\item Conceptually, describe the action of $D$. \\
$D$ stretches each $e_i$th component by a factor of $\lambda_i$. \\ 
\item Conceptually, describe the action of $P^{-1}DP$ \\
First, $P$ transforms vectors from the canonical basis to the basis of vectors $v_1,...,v_n$ . $D$ then stretches the  $v_1,...,v_n$ \textbf{components} by $\lambda_1,...,\lambda_n$. 
\end{solution}
\end{enumerate}
\end{frame}
\end{comment}

\end{document}
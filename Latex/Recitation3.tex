\documentclass{beamer}
\usepackage{amsthm}
\usepackage[utf8]{inputenc}
\usetheme{madrid}
\usepackage{../latex_style/packages}
\usepackage{../latex_style/notations}
\usepackage{outlines}

\def\labelenumi{\theenumi}

\setbeamertemplate{itemize items}[default]
\setbeamertemplate{enumerate items}[default]

\newcommand\Fonteight{\fontsize{8}{9.6}\selectfont}

%Information to be included in the title page:
\title{Recitation 3 }
\author{Alex Dong}
\institute{CDS, NYU}
\date{Fall 2020}


\makeatletter
\setbeamertemplate{footline}{
  \leavevmode%
  \hbox{%
  \begin{beamercolorbox}[wd=.4\paperwidth,ht=2.25ex,dp=1ex,center]{author in head/foot}%
    \usebeamerfont{author in head/foot}\insertshortauthor\expandafter\ifblank\expandafter{\beamer@shortinstitute}{}{~~(\insertshortinstitute)}
  \end{beamercolorbox}%
  \begin{beamercolorbox}[wd=.3\paperwidth,ht=2.25ex,dp=1ex,center]{title in head/foot}%
    \usebeamerfont{title in head/foot}\insertshorttitle
  \end{beamercolorbox}%
  \begin{beamercolorbox}[wd=.3\paperwidth,ht=2.25ex,dp=1ex,right]{date in head/foot}%
    \usebeamerfont{date in head/foot}\insertshortdate{}\hspace*{2em}
    \insertframenumber{} / \inserttotalframenumber\hspace*{2ex} 
  \end{beamercolorbox}}%
  \vskip0pt%
}
\makeatother


\begin{document}
%1
\frame{\titlepage} 
%2
\begin{frame}
\frametitle{Rank Nullity Theorem}
\begin{theorem}[Rank-nullity theorem]
	Let $L: \R^m \to \R^n$ be a linear transformation. Then
	$$
	\rank(L) + \dim(\Ker(L)) = m.
	$$
\end{theorem}
One of the most important theorems in linear algebra.\\
You should be able to state and prove this theorem (with no notes).
\end{frame}


\begin{frame}
\frametitle{Symmetric Matrices: That's cute!}
Symmetric Matrices are not just "cute"...\\
They are actually DEEPLY LINKED to many topics in linear algebra.\\
Orthogonal Projections(Next lecture) are symmetric matrices.\\
Spectral theorem - "eigenvectors and eigenvalues of symmetric matrices are really special".\\
PCA: Covariance matrix is symmetric
Concavity: Hessian Matrix (matrix of double derivatives) is symmetric

 
But, we will see most of this later.\\
For now, just trust me :)
\end{frame}







\end{document}
\documentclass{beamer}
\usepackage{amsthm}
\usepackage[utf8]{inputenc}
\usetheme{Madrid}
\usepackage{../latex_style/packages_old}
\usepackage{../latex_style/notations_old}
\usepackage{outlines}
\usepackage{enumitem}
\renewenvironment{itemize}
\usefonttheme{serif}
\def\labelenumi{\theenumi}
\usefonttheme{serif}
\usepackage{comment}

%\setbeamertemplate{itemize items}[default]
%\setbeamertemplate{enumerate items}[default]

% define smaller font command
\newcommand*{\horzbar}{\rule[.5ex]{2.5ex}{0.5pt}}
\newcommand\fonteight{\fontsize{8}{9.6}\selectfont}
\newcommand\fontten{\fontsize{10}{1.2}\selectfont}

%define enumerate with periods
\renewenvironment{enumerate}%
{\begin{list}{\arabic{enumi}.}%  \langle ------ dot here
      {\setlength{\leftmargin}{2.5em}%
       \setlength{\itemsep}{-\parsep}%
       \setlength{\topsep}{-\parskip}%%
       \usecounter{enumi}}%
 }{\end{list}}
%define itemize with arrows
\renewenvironment{itemize}%
{\begin{list}{$\blacktriangleright$}%  \langle ------ dot here
      {\setlength{\leftmargin}{2.5em}%
       \setlength{\itemsep}{-\parsep}%
       \setlength{\topsep}{-\parskip}%%
       \usecounter{enumi}}%
 }{\end{list}}

%Information to be included in the title page:
\title{Recitation 11}
\author{Alex Dong}
\institute{CDS, NYU}
\date{Fall 2020}


\makeatletter
\setbeamertemplate{navigation symbols}{}
\setbeamertemplate{footline}{
  \leavevmode%
  \hbox{%
  \begin{beamercolorbox}[wd=.4\paperwidth,ht=2.25ex,dp=1ex,center]{author in head/foot}%
    \usebeamerfont{author in head/foot}\insertshortauthor\expandafter\ifblank\expandafter{\beamer@shortinstitute}{}{~~(\insertshortinstitute)}
  \end{beamercolorbox}%
  \begin{beamercolorbox}[wd=.3\paperwidth,ht=2.25ex,dp=1ex,center]{title in head/foot}%
    \usebeamerfont{title in head/foot}\insertshorttitle
  \end{beamercolorbox}%
  \begin{beamercolorbox}[wd=.3\paperwidth,ht=2.25ex,dp=1ex,right]{date in head/foot}%
    \usebeamerfont{date in head/foot}\insertshortdate{}\hspace*{2em}
    \insertframenumber{} / \inserttotalframenumber\hspace*{2ex} 
  \end{beamercolorbox}}%
  \vskip0pt%
}
\setbeamertemplate{navigation symbols}{}
\makeatother

\begin{document}
%1
\frame{\titlepage} 
%2

\begin{frame}
\frametitle{Questions: Unconstrained Optimization}
Let $f: \R^n \rightarrow \R$ be twice differentiable. Let $x,h\in \R^n$.\\
 True or False.
\begin{enumerate}
\item If $f$ is convex, then $f(x+h)>f(x)+\langle \nabla f(x),h \rangle$.
\item If $f$ is strictly convex, then $f(x+h)>f(x)+\langle \nabla f(x),h \rangle$.
\item If $f$ is strongly convex, then $f(x+h)>f(x)+\langle \nabla f(x),h \rangle$.
\item If $f$ is convex, then $f$ cannot have saddle points.
\item If $\nabla f(x) = 0$, then $x$ is a local minimum of $f$.
\item If $\nabla f(x) = 0$ and $H_f(x) \succeq 0$, then $x$ is a local minimum of $f$.
\item If $\nabla f(x) = 0$ and $H_f(x) \succ 0$, then $x$ is a local minimum of $f$.
\item If $\nabla f(x) = 0$ and $f$ is convex, then $x$ is a local minimum of $f$.
\end {enumerate}
\end{frame}

\begin{frame}
\frametitle{Solutions: Unconstrained Optimization}
Let $f: \R^n \rightarrow \R$ be twice differentiable. Let $x,h\in \R^n$.\\
 True or False.
\begin{solution}
\begin{enumerate}
\item If $f$ is convex, then $f(x+h)>f(x)+\langle \nabla f(x),h \rangle$.
False, consider $f(x) = <x,w>$ for some $w\in\R^n$. 
\item If $f$ is strictly convex, then $f(x+h)>f(x)+\langle \nabla f(x),h \rangle$.\\
True. (From Lec 9)
\item If $f$ is strongly convex, then $f(x+h)>f(x)+\langle \nabla f(x),h \rangle$.\\
True, because strongly convex implies strictly convex
\item If $f$ is convex, then $f$ cannot have saddle points.\\
True. If f had a saddle point, we could draw a chord below $f$ on the "negative" side of the saddle.
\end{enumerate}
\end{solution}
\end{frame}

\begin{frame}
\frametitle{Solutions: Unconstrained Optimization}
Let $f: \R^n \rightarrow \R$ be twice differentiable. Let $x,h\in \R^n$.\\
 True or False.
\begin{solution}
\begin{enumerate}
\item[5.] If $\nabla f(x) = 0$, then $x$ is a local minimum of $f$.\\
False! Consider $f(x) = -x^2$.
\item[6.] If $\nabla f(x) = 0$ and $H_f(x) \succeq 0$, then $x$ is a local minimum of $f$.\\
False, consider $f(x) = -x^4$.
\item[7.] If $\nabla f(x) = 0$ and $H_f(x) \succ 0$, then $x$ is a local minimum of $f$.\\
True, this is exactly the condition we need for local minimums!
\item[8.] If $\nabla f(x) = 0$ and $f$ is convex, then $x$ is a local minimum of $f$.\\
True, $f$ being convex implies the Hessian is PSD everywhere.
\end{enumerate}
\end{solution}
\end{frame}

\begin{frame}
\frametitle{Questions: Constrained Optimization}
Let $f: \R^n \rightarrow \R$ be twice differentiable. Let $g(x) \leq 0$, and $h(x)=0$ be two constraints. Let $x\in \R^n$ be in the feasible set.\\
True or False.
\begin{enumerate}
\item If $f,g,h$ are convex, and $\nabla f(x) = 0$, then $x$ is a local minimum.
\item If $f,g,h$ are convex, and $\nabla f(x) = 0$, then $x$ is a global minimum. 
\item If $x$ is a local minimum in the feasible set, then $\nabla f(x) = 0$.
\item (Ignoring $h$) If $x$ is a local minimum in the feasible set, and $g(x) < 0$, then $\nabla f(x) = 0$. 
\end{enumerate}
\end{frame}

\begin{frame}
\frametitle{Solutions : Constrained Optimization}
Let $f: \R^n \rightarrow \R$ be twice differentiable. Let $g(x) \leq 0$, and $h(x)=0$ be two constraints. Let $x\in \R^n$ be in the feasible set.\\
True or False.
\begin{solution}
\begin{enumerate}
\item If $f,g,h$ are convex, and $\nabla f(x) = 0$, then $x$ is a local minimum. \\
True! Since $x$ is in the feasible set and $\nabla f(x) = 0$ , and $f$ is convex, then $x$ is a local minimum.

\item If $f,g,h$ are convex, and $\nabla f(x) = 0$, then $x$ is a global minimum. \\
True! Local minimums are global minimums since $f$ is convex.

\item If $x$ is a local minimum in the feasible set, then $\nabla f(x) = 0$. \\
False, we could be limited by a constraint. Consider $f(x) = x^2$, $0 \geq g(x) = x+1$.  

\item (Ignoring $h$) If $x$ is a local minimum in the feasible set, and $g(x) < 0$, then $\nabla f(x) = 0$. \\
True! Since our constraint is not active, the lagrange multiplier is 0.
\end{enumerate}
\end{solution}
\end{frame}



\begin{frame}
\frametitle{Question : Constrained Optimization}
Let $f(x,y) = x^2-10x + y^2 - 2y$. Let $h(x,y) = x^2+y^2 = 26$\begin{enumerate}
\item Find the minimum of $f$ constrained by $h(x,y) = 0$.
\end{enumerate}
\end{frame}


\begin{frame}
\frametitle{Solution : Constrained Optimization}
Let $f(x,y) = x^2-10x + y^2 - 2y$. Let $h(x,y) = x^2+y^2 = 26$
\begin{solution}
Since h is an active constraint, we can solve this by considering the following system of equations.\\
\quad $\nabla f(x,y) + \lambda \nabla g(x,y) = 0$ and  $h(x,y) = 0$. \\
Now, 
$\nabla f(x,y) = 
\begin{bmatrix}
2x - 10  \\
2y - 4 \\
\end{bmatrix}$, 
$\nabla h(x,y) = 
\begin{bmatrix}
2x \\
2y \\
\end{bmatrix}$. \\
This gives us a system of three equations and three unknowns.\\
\quad $2x - 10 + \lambda 2 x = 0$ \\
\quad $2y - 2  + \lambda 2 y = 0$ \\
\quad $x^2+y^2 = 26$ \\
Solving this system of equations gives two combinations of $(\lambda, x, y)$\\
$(\lambda, x, y)  =  (0,5,1),$ and $(\lambda, x, y) = (-2,-5,-1)$.

Note: $\nabla^2 f(x,y) = H_f(x,y) = 
\begin{bmatrix}
2 & 0 \\
0 & 2 \\
\end{bmatrix}$, so both points are local minima. f(5,1) = -28, f(-5,-1) = 79 
\end{solution}
\end{frame}





\end{document}
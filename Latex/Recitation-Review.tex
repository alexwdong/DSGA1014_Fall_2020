\documentclass{beamer}
\usepackage{amsthm}
\usepackage[utf8]{inputenc}
\usetheme{Madrid}
\usepackage{../latex_style/packages_old}
\usepackage{../latex_style/notations_old}
\usepackage{outlines}
\usepackage{enumitem}
\renewenvironment{itemize}
\usefonttheme{serif}
\def\labelenumi{\theenumi}
\usefonttheme{serif}
\usepackage{comment}

%\setbeamertemplate{itemize items}[default]
%\setbeamertemplate{enumerate items}[default]

% define smaller font command
\newcommand*{\horzbar}{\rule[.5ex]{2.5ex}{0.5pt}}
\newcommand\fonteight{\fontsize{8}{9.6}\selectfont}
\newcommand\fontten{\fontsize{10}{1.2}\selectfont}

%define enumerate with periods
\renewenvironment{enumerate}%
{\begin{list}{\arabic{enumi}.}%  \langle ------ dot here
      {\setlength{\leftmargin}{2.5em}%
       \setlength{\itemsep}{-\parsep}%
       \setlength{\topsep}{-\parskip}%%
       \usecounter{enumi}}%
 }{\end{list}}
%define itemize with arrows
\renewenvironment{itemize}%
{\begin{list}{$\blacktriangleright$}%  \langle ------ dot here
      {\setlength{\leftmargin}{2.5em}%
       \setlength{\itemsep}{-\parsep}%
       \setlength{\topsep}{-\parskip}%%
       \usecounter{enumi}}%
 }{\end{list}}

%Information to be included in the title page:
\title{Recitation (Review Week)}
\author{Alex Dong}
\institute{CDS, NYU}
\date{Fall 2020}


\makeatletter
\setbeamertemplate{navigation symbols}{}
\setbeamertemplate{footline}{
  \leavevmode%
  \hbox{%
  \begin{beamercolorbox}[wd=.4\paperwidth,ht=2.25ex,dp=1ex,center]{author in head/foot}%
    \usebeamerfont{author in head/foot}\insertshortauthor\expandafter\ifblank\expandafter{\beamer@shortinstitute}{}{~~(\insertshortinstitute)}
  \end{beamercolorbox}%
  \begin{beamercolorbox}[wd=.3\paperwidth,ht=2.25ex,dp=1ex,center]{title in head/foot}%
    \usebeamerfont{title in head/foot}\insertshorttitle
  \end{beamercolorbox}%
  \begin{beamercolorbox}[wd=.3\paperwidth,ht=2.25ex,dp=1ex,right]{date in head/foot}%
    \usebeamerfont{date in head/foot}\insertshortdate{}\hspace*{2em}
    \insertframenumber{} / \inserttotalframenumber\hspace*{2ex} 
  \end{beamercolorbox}}%
  \vskip0pt%
}
\setbeamertemplate{navigation symbols}{}
\makeatother

\begin{document}
%1
\frame{\titlepage} 
%2

\begin{frame}
\frametitle{Review day: Linear Regression and SVD}
\begin{itemize}
\item Your Questions
\item Linear Regression Review
\item SVD Review
\end{itemize}


\end{frame}


\begin{frame}
\frametitle{Question: Linear Regression Review 1}

\begin{itemize}
\item Goal is to minimize $\ell = \|X\beta - y\|_2^2$ w.r.t $\beta$
\begin{itemize}
\item $\ell$ is loss
\item $X$ is your data/design matrix
\item $\beta$ is the coefficients that transform $X$ into $y$
\end{itemize}
\end{itemize}
 
Let $x_1,..,x_n \in \R^d$, $y_1,..,y_n, \in\R$. Let $\beta \in\R^d$\\
\begin{enumerate}
\item Recall that $X$ is defined to have $x_1,..,x_n$ as the rows of $X$, and $y$ is defined to have $y_1,...,y_n$ as its entries. \\
Show that $\|X\beta - y\|_2^2 = \sum_{i=1}^n (y_i - \langle x_i,\beta \rangle)^2$

\item Notice that when $x_i = \vec{0}$, $\langle x_i, \beta \rangle = 0$. Suppose we wanted to add an `intercept', or `bias' $\beta_0\in \R$,
so that we are trying to minimize $\|X\beta - y\|_2^2 = \sum_{i=1}^n (y_i - \langle x_i,\beta \rangle-\beta_0)^2$.\\
How can we modify our definitions of $X,y,\beta$ to easily do this?
\end{enumerate}


\end{frame}


\begin{frame}
\frametitle{Solutions: Linear Regression Review 1}
\begin{solution}
\begin{enumerate}
\item Recall that $X$ is defined to have $x_1,..,x_n$ as the rows of $X$, and $y$ is defined to have $y_1,...,y_n$ as its entries. \\
Show that $\|X\beta - y\|_2^2 = \sum_{i=1}^n (y_i - \langle x_i,\beta \rangle)^2$ \\
Use Inner product method of matrix multiplication
\item Notice that when $x_i = \vec{0}$, $\langle x_i, \beta \rangle = 0$. Suppose we wanted to add an `intercept', or `bias' $\beta_0\in \R$,
so that we are trying to minimize $\|X\beta - y\|_2^2 = \sum_{i=1}^n (y_i - \langle x_i,\beta \rangle-\beta_0)^2$.\\
How can we modify our definitions of $X,y,\beta$ to easily do this?\\
Append/prepend a column of $1$'s to $X$, and append/prepend $\beta_0$.
\end{enumerate}
\end{solution}
\end{frame}

\begin{frame}
\frametitle{Questions: Ridge Regression and Multicollinearity}
Let $X \in \R^{n\times d}$, $n>d$, and \textit{not have full rank}. ($X$ is a data matrix) \\
Recall that the OLS solution is $\hat{x} = (X^TX)^{-1}X^Ty$.
\begin{enumerate}
\item Since $X$ is not full rank, what does this say about the features?
\item What is the issue with the OLS solution?
\item The ridge regression solution is given by $(X^TX+\lambda Id_d)^{-1}X^Ty$. How does this fix the issue?
\item Suppose that $X$ has SVD $X=U\Sigma V^T$, and $X$ has singular values $\sigma_1,...,\sigma_d$. What are the eigenvalues of 
$X^TX + \lambda Id_d$?
\item How does increasing $\lambda$ affect the condition number of $(X^TX + \lambda Id_d)$?
\end{enumerate}
\end{frame}


\begin{frame}
\frametitle{Solutions: Ridge Regression and Multicollinearity}
Let $X \in \R^{n\times d}$, $n>d$, and \textit{not have full rank}. ($X$ is a data matrix) \\
Recall that the OLS solution is $\hat{x} = (X^TX)^{-1}X^Ty$.
\begin{solution}
\begin{enumerate}
\item Since $X$ is not full rank, what does this say about the features?\\
Columns of $X$ are not linearly independent, so some of the features can be perfectly explained by other features.
\item What is the issue with the OLS solution? \\
Since $X$ does not have full rank, $X^TX$ doesn't have full rank and is not invertible. So the OLS solution is not well-defined.
\item The ridge regression solution is given by $(X^TX+\lambda Id_d)^{-1}X^Ty$. How does this fix the issue? \\
Adding $\lambda Id_d$ to $X^TX$ shifts its eigenvalues up, which makes $(X^TX+\lambda Id_d)$ invertible.
\end{enumerate}
\end{solution}
\end{frame}

\begin{frame}
\frametitle{Solutions: Ridge Regression and Multicollinearity}
Let $X \in \R^{n\times d}$, $n>d$, and \textit{not have full rank}. ($X$ is a data matrix) \\
Recall that the OLS solution is $\hat{x} = (X^TX)^{-1}X^Ty$.
\begin{solution}
\begin{enumerate}
\item[4.] Suppose that $X$ has SVD $X=U\Sigma V^T$, and $X$ has singular values $\sigma_1,...,\sigma_d$. What are the eigenvalues of 
$X^TX + \lambda Id_d$? \\
Note that $X^TX = V\Sigma^T \Sigma V^T$\\
Eigvals of $X^TX$: $\sigma_1^2,...,\sigma_d^2$, (Note: $X$ isn't full rank, so $\sigma_d = 0$) \\
Eigvals of $X^TX+\lambda Id_d$: $\sigma_1^2 + \lambda ,..., \sigma_d^2+\lambda$.\\
\item[5.] How does increasing $\lambda$ affect the condition number of $(X^TX + \lambda Id_d)$ vs $X^TX$?\\
Condition number of $X^TX = \frac{\sigma_1^2}{\sigma_d^2} = \infty$ \\
Condition number of $(X^TX + \lambda Id_d) = \frac{\sigma_1^2+\lambda}{\sigma_d^2+\lambda}$\\
Furthermore, for $\lambda_1 > \lambda_2$, we get the relationship \\
\quad $ \frac{\sigma_1^2+\lambda_1}{\sigma_d^2+\lambda_1} <  \frac{\sigma_1^2+\lambda_2}{\sigma_d^2+\lambda_2}$
\end{enumerate}
\end{solution}
\end{frame}



\begin{frame}
\frametitle{Questions: More SVD}
Let $A\in \R^{n\times m}$, where $n>m$, have SVD $A=U\Sigma V^T$.\\ 
Let $e_{i,m}, e_{i,n}$ denote the $i$th standard basis vector in $\R^m$, and $\R^n$.\begin{enumerate}

\item Give basis transformations for $U$, $\Sigma$, $V^T$.\\
(Your answer should look like $T(a) = b$. where you select convenient $a$'s that form a basis for the origin space of T.  Be careful about the dimensions w/ $\Sigma$!)

\medskip
\item Let $x\in \R^m$ s.t $x = \sum_{i=1}^m\alpha_i v_i$, where $v_i$ are the columns of $V$.\\
 and write the expressions for $V^Tx$, $\Sigma V^Tx$, and $U\Sigma V^Tx$.
\medskip
\item Use this to show $Im(AA^T) = Im(A)$
\end{enumerate}
\end{frame}


\begin{frame}
\frametitle{Solutions 1: More SVD}
Let $A\in \R^{n\times m}$, where $n>m$, have SVD $A=U\Sigma V^T$.\\ 
Let $e_{i,m}, e_{i,n}$ denote the $i$th standard basis vector in $\R^m$, and $\R^n$.
\begin{enumerate}
\item Give basis transformations for $U$, $\Sigma$, $V^T$.\\
(Your answer should look like $T(a) = b$. where you select convenient $a$'s that form a basis for the origin space of T.  Be careful about the dimensions w/ $\Sigma$!)
\begin{solution}
$V^T(v_i) = e_{i,m}$ \quad for $i\in\{1,...,m\}$ \\
$\Sigma(e_{i,m}) = \sigma_i e_{i,n}$ \quad for $i \in \{1, ..., m\}$ (Note the $m$ here) \\
$U(e_{i,n}) = u_{i}$ \quad for $i \in \{1, ..., n\}$ \\
\end{solution}
\end{enumerate}
\end{frame}



\begin{frame}
\frametitle{Solutions 2: More SVD}
Let $A\in \R^{n\times m}$, where $n>m$, have SVD $A=U\Sigma V^T$.\\ 
Let $e_{i,m}, e_{i,n}$ denote the $i$th standard basis vector in $\R^m$, and $\R^n$.
\begin{enumerate}
\item[2.] Let $x\in \R^m$ s.t $x = \sum_{i=1}^m\alpha_i v_i$, where $v_i$ are the columns of $V$.\\ Write the expressions for $V^Tx$ $\Sigma V^Tx$, and $U\Sigma V^Tx$.
\begin{solution}
Let $e_{i,m}$ denote the $i$th standard basis vector in $\R^m$. \\
$x = \sum_{i=1}^m\alpha_i v_i$\\
$V^Tx = \sum_{i=1}^m\alpha_i e_{i,m}$ \\
$\Sigma V^Tx = \sum_{i=1}^{min(m,n)} \sigma_i \alpha_i e_{i,n}$ \\
$U \Sigma V^Tx = \sum_{i=1}^{min(m,n)} \sigma_i \alpha_i u_i$ \\
\end{solution}
\end{enumerate}
\end{frame}

\begin{frame}
\frametitle{Solutions 3: More SVD}
Let $A\in \R^{n\times m}$, where $n>m$, have SVD $A=U\Sigma V^T$.\\ 
Let $e_{i,m}, e_{i,n}$ denote the $i$th standard basis vector in $\R^m$, and $\R^n$.
\begin{enumerate}
\item[3.] Use this to show $Im(AA^T) = Im(A)$
\begin{solution}
Let $x \in \R^m, x = \sum_{i=1}^m \alpha_i v_i$. Then $Ax =  \sum_{i=1}^m \alpha_i \sigma_i u_i$. \\
Let $y \in \R^n, y = \sum_{i=1}^n \beta_i u_i$. Then $AA^Ty = U\Sigma\Sigma^TU^Ty = \sum_{i=1}^m \beta_i \sigma_i^2 u_i$.\\

We show $Im(A) \subset Im(AA^T)$\\
Now, let $p \in \Im(A) \subset\R^n$.\\
$z$ must be of the form $p = \sum_{i=1}^m \gamma_i u_i$. \\
Notice that for $q\in \R^m$, where\\
\quad  $q = \sum_{i=1}^m c_i v_i$,\quad and \quad $c_i = \frac{\gamma_i}{\sigma_i}$ if $\gamma_i\neq 0$, $c_i = 0$ if $\gamma_i=0$\\
We get $Aq = p$.
Similarly, if we let $r\in \R^n$ where\\
\quad $r =  \sum_{i=1}^m d_i v_i$, \quad and \quad $d_i = \frac{\gamma_i}{\sigma_i^2}$ if $\gamma_i\neq 0$, $d_i = 0$ if $\gamma_i=0$\\
We get $AA^Tr = p$.\\
So, we have found a vector r, s.t $AA^Tr = p\in Im(A$). Then $p \in \Im(AA^T)$, and $Im(A)\subset Im(A^T).$
\end{solution}
\end{enumerate}
\end{frame}



\end{document}
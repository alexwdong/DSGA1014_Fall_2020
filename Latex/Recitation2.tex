
\documentclass[table]{beamer}
\usepackage{xcolor}

\usepackage{amsthm}
\usepackage[utf8]{inputenc}
\usetheme{Madrid}
\usepackage{../latex_style/packages}
\usepackage{../latex_style/notations}
\usepackage{outlines}
\usepackage{enumitem}

\usefonttheme{serif}
\def\labelenumi{\theenumi}

%\setbeamertemplate{itemize items}[default]
%\setbeamertemplate{enumerate items}[default]

% define smaller font command
\newcommand\Fonteight{\fontsize{8}{9.6}\selectfont}
%define enumerate with periods
\renewenvironment{enumerate}%
{\begin{list}{\arabic{enumi}.}% <------ dot here
      {\setlength{\leftmargin}{2.5em}%
       \setlength{\itemsep}{-\parsep}%
       \setlength{\topsep}{-\parskip}%%
       \usecounter{enumi}}%
 }{\end{list}}

%Information to be included in the title page:
\title{Recitation 2}
\author{Alex Dong}
\institute{CDS, NYU}
\date{Fall 2020}


\makeatletter
\setbeamertemplate{navigation symbols}{}
\setbeamertemplate{footline}{
  \leavevmode%
  \hbox{%
  \begin{beamercolorbox}[wd=.4\paperwidth,ht=2.25ex,dp=1ex,center]{author in head/foot}%
    \usebeamerfont{author in head/foot}\insertshortauthor\expandafter\ifblank\expandafter{\beamer@shortinstitute}{}{~~(\insertshortinstitute)}
  \end{beamercolorbox}%
  \begin{beamercolorbox}[wd=.3\paperwidth,ht=2.25ex,dp=1ex,center]{title in head/foot}%
    \usebeamerfont{title in head/foot}\insertshorttitle
  \end{beamercolorbox}%
  \begin{beamercolorbox}[wd=.3\paperwidth,ht=2.25ex,dp=1ex,right]{date in head/foot}%
    \usebeamerfont{date in head/foot}\insertshortdate{}\hspace*{2em}
    \insertframenumber{} / \inserttotalframenumber\hspace*{2ex} 
  \end{beamercolorbox}}%
  \vskip0pt%
}
\setbeamertemplate{navigation symbols}{}
\makeatother

\begin{document}
%----Slide 1-----------------------

\frame{\titlepage} 

%----Slide 2-----------------------

\begin{frame}
\frametitle{Some Etymology...}

\begin{definition}[Linearity: Wikipedia]
The property of a mathematical relationship (function) that can be graphically represented as a straight line.
\end{definition}

\begin{definition}[Algebra: Wikipedia]
The study of mathematical symbols and the rules for manipulating these symbols.
\end{definition}
Linear algebra is the study of manipulating letters/symbols which are used to represent linear transformations.
\end{frame}

%----Slide 3-----------------------

\begin{frame}
\frametitle{Concept Review: Linear Transformations}

\begin{definition}[Linear Transformation]	        
		A function $L: \R^m \to \R^n$ is linear if
		\Fonteight
		
	\begin{enumerate}
		\item for all $v \in \R^m$ and all $\alpha \in \R$ we have $L(\alpha v) = \alpha L(v)$ and
		\item for all $v,w \in \R^m$ we have $L(v + w) = L(v) + L(w)$.
	\end{enumerate}
\end{definition}

One of the most important (but boring) theorems in Linear Algebra.

\begin{theorem}[Matrix Representation Theorem]
All linear transformations can be represented as matrices; all matrices represent linear transformations.
\end{theorem}
Think about which framework to use in your proofs!
\end{frame}

%----Slide 4-----------------------

\begin{frame}
\frametitle{A Note about Gaussian Elimination}

Gaussian elimination is a procedure to calculate the solutions of a matrix equation.\\
We will not cover this in the course, but you should at least be familiar with it.\\
If you've already studied it in previous courses, that should be enough.\\
If this is the first time you've heard this, then please do some light studying to familiarize yourself with the process.\\

\end{frame}


%----Slide 5-----------------------

\begin{frame}
\frametitle{Matrix Notation}
A linear transformation $ T: \R^n \to \R^m$ is represented by a $m \times n $ matrix
which is an element of $\R^{m\times n}$. (!!Note the order!!)\\
T = \bordermatrix{ &  & n &  \cr
       & T_{1,1} & ... & T_{1,n} \cr
      m & \vdots & \ddots & \vdots \cr
       & T_{m,1} & \dots & T_{m,n} }
\\
\bigskip
This matrix has $m$ rows and $n$ columns. \\
$T_{i,j}$ represents the value of the entry in the $i$th row and $j$th column.
\end{frame}

%----Slide 6-----------------------

\begin{frame}
\frametitle{Linear Transformations and Subspaces}
Linear transformations are fundamentally connected to subspaces. How?
We will spend a lot of time on investigating the \textit{action} of a linear transformation 
on subspaces.\\
\medskip
Key questions in linear algebra: 
\begin{itemize}
\item What does a linear transformation do to 1-dimensional subspaces?
\item What does a linear transformation to do a n-dimensional subspace?
\end{itemize}

\end{frame}

%----Slide 7-----------------------

\begin{frame}
\frametitle{Questions 1: Linear Transformations}
Which of the following functions are linear?  If the function is
linear, what is the kernel?
  \begin{enumerate}
  \item $f_1:\R^2\to\R^2$ such that $f_1(a,b) = (2a,a+b)$
  \item $f_2:\R^2\to\R^3$ such that $f_2(a,b) = (a+b,2a+2b,0)$
  \item $f_3:\R^2\to\R^3$ such that $f_3(a,b) = (2a,a+b,1)$
  \item $f_4:\R^2\to\R$ such that $f_4(a,b) = \sqrt{a^2+b^2}$
  \item $f_5:\R\to\R$ such that $f_5(x) = 5x+3$
  \end{enumerate}
\end{frame}

%----Slide 8-----------------------

\begin{frame}
\frametitle{Solutions 1: Linear Transformations}
Which of the following functions are linear?  If the function is
linear, what is the kernel?
  \begin{enumerate}
  \item $f_1:\R^2\to\R^2$ such that $f_1(a,b) = (2a,a+b)$
  \item $f_2:\R^2\to\R^3$ such that $f_2(a,b) = (a+b,2a+2b,0)$
  \item $f_3:\R^2\to\R^3$ such that $f_3(a,b) = (2a,a+b,1)$
  \item $f_4:\R^2\to\R$ such that $f_4(a,b) = \sqrt{a^2+b^2}$
  \item $f_5:\R\to\R$ such that $f_5(x) = 5x+3$
  \end{enumerate}
\begin{solution}
\begin{enumerate}
  \item Linear, Kernel is $\{0\}$.
  \item Linear, Kernel is $\{(c,-c) : c\in\R\}$.
  \item Not linear, $f_3(0,0)=(0,0,1)$.
  \item Not linear, $f_4(1,0)+f_4(0,1)=2$ and $f_4(1,1)=\sqrt{2}$.
  \item Not linear, $f_5(0)=3$.
  \end{enumerate}
\end{solution}
\end{frame}

%----Slide 9-----------------------

\begin{frame}
\frametitle{Questions 2: Matrix Manipulation}
Let 
$A = \begin{bmatrix}
			5 & 0 & 0  \\
			0 & 0 & 1  \\
			0 & 1 & 0   \\
			\end{bmatrix}$
$B = \begin{bmatrix}
			1 & 0 & 0 & 2 \\
			0 & 1 & 0 & 3 \\
			0 & 0 & 1 & 4 \\
			\end{bmatrix}$
$C = \begin{bmatrix}
			0 & 0 & 0 & 0  \\
			2 & 1 & 0 & 0 \\
			1 & 0 & 0 & 1  \\
			0 & 0 & 1 & 0  \\
			\end{bmatrix}$
  \begin{enumerate}
  \item Calculate AB
  \item Calculate BC
  \item What does A do to B?
  \item What does C do to B?
  \end{enumerate}
\end{frame}

%----Slide 10-----------------------

\begin{frame}
\frametitle{Solutions 2: Matrix Manipulation}
\begin{solution}
\begin{enumerate}
  \item $AB = \begin{bmatrix}
			5 & 0 & 0 & 10  \\
			0 & 0 & 1 & 4   \\
			0 & 1 & 0 & 3   \\
			\end{bmatrix}$
  \item $BC = \begin{bmatrix}
			0 & 0 & 2 & 0  \\
			2 & 1 & 3 & 0   \\
			1 & 0 & 4 & 1  \\
			\end{bmatrix}$
  \item Five times first row, switch second and third row
  \item First column becomes twice the second column plus one times third column, second column stays the same, switch 3rd and fourth columns.
  \
  \end{enumerate}
\end{solution}
\end{frame}

%----Slide 11-----------------------

\begin{frame}
\frametitle{Questions 1: Invertibility}
Let $S \in \R^{n\times n}$, $T \in \R^{n\times k}$ and $U \in \R^{k \times k}$. \\
Let $S$ and $U$ be invertible.\\ 
\begin{enumerate}
\item Prove that $Ker(S) = 0$.\\
\end{enumerate}

Now, prove or give a counter example to the following statements:
\begin{enumerate}
\item[2.]{$Ker(T) = Ker(TU)$}
\item[3.]{$Ker(ST) = Ker(T)$}
\end{enumerate}

\end{frame}

%----Slide 12-----------------------

\begin{frame}
\frametitle{Solutions 3: Invertibility}
Let $S \in \R^{n\times n}$, $T \in \R^{n\times k}$ and $U \in \R^{k \times k}$. \\
Let $S$ and $U$ be invertible.\\ 
\begin{solution}
\Fonteight
\begin{enumerate}
	\item Prove that $Ker(S) = 0$.\\
    We prove by contradiction. \\
    Suppose that $Ker(S)\neq{0}$. Then $\exists x\neq 0$ s.t $Sx = 0$.\\
    Now, consider $S^{-1}Sx$. \\
    $(S^{-1}S)x = Ix = x$,\\
    and $S^{-1}(Sx)=0$. \\
    We have reached a contradiction, so $Ker(S) = 0$ 
\end{enumerate}
\end{solution}
\end{frame}

%----Slide 13-----------------------

\begin{frame}
\frametitle{Solutions 3: Invertibility}
\begin{solution}
\begin{enumerate}
\item[2.]{$Ker(T) = Ker(TU)$. \textbf{False}}\\
	Consider $T = \begin{bmatrix}
			1 & 0 \\
			0 & 0 \\
			\end{bmatrix}
			U = \begin{bmatrix}
			0 & 1 \\
			1 & 0 \\
			\end{bmatrix}$\\
	$ Ker(T) = \{\begin{bmatrix}
			0 \\
			y \\
			\end{bmatrix}\ |\ y\in\R\}$.\\
	$\Ker(TU) = \{\begin{bmatrix}
			x \\
			0 \\
			\end{bmatrix}\ |\ x\in\R\}$

\item[3.]{$Ker(ST) = Ker(T)$}. \textbf{True}\\
We'll show that $Ker(ST)\subset Ker(T).$\\
Let $x \in Ker(ST)$.\\
So, $STx = 0$.\\
Since $S$ is invertible, then $Ker(S) = 0$.\\
Therefore, $Tx=0$, and $x\in Ker(T)$.\\
$Ker(T)\subset Ker(ST)$ is straightforward.
\end{enumerate}
\end{solution}
\end{frame}


\begin{frame}
\frametitle{Matrix Multiplication Mechanics: Inner Products}
(We haven't defined inner product yet)\\
Let $A\in \R^{n\times k}$, $B \in \R^{k\times m}$\\
Rows of first matrix ``line up" with columns of the second matrix.\\
\Fonteight
$\left[\begin{matrix}
\rowcolor{red!20}    a_{1,1} & \hdots & a_{1,k} \\ 
\rowcolor{blue!20}   a_{2,1} & \hdots & a_{2,k} \\ 
                      \vdots & \hdots & \vdots \\
\rowcolor{green!20}  a_{n-1,1} & \hdots & a_{n-1,k} \\ 
\rowcolor{yellow!20} a_{n,1} & \hdots & a_{n,k} \\ 
\end{matrix}\right]$
$\left[\begin{array}{>{\columncolor{gray!20}}ccccc}
b_{1,1} & b_{1,2} &\hdots & b_{1,m-1} & b_{1,m} \\ 
\vdots & \vdots & \hdots & \vdots & \vdots \\
b_{k,1} &  b_{k,2} &\hdots & b_{k,m-1} & b_{k,m} 
\end{array}\right]$
$=
\left[\begin{array}{ccccc}
{\cellcolor{red!20}}    \sum_{i=0}^k a_{1,i}b_{i,1} & \hdots & \hdots & & \\
{\cellcolor{blue!20}}   \sum_{i=0}^k a_{2,i}b_{i,1} & \hdots & \hdots& & \\
                        \vdots & \hdots & \hdots & & \\
{\cellcolor{green!20}}  \sum_{i=0}^k a_{n-1,i}b_{i,1} & \hdots& \hdots& & \\
{\cellcolor{yellow!20}} \sum_{i=0}^k a_{n,i}b_{i,1} & \hdots& \hdots& & \\
\end{array}\right]
$
\end{frame}

\begin{frame}
\frametitle{Matrix Multiplication Mechanics: Inner Products}
(We haven't defined inner product yet)\\
Let $A\in \R^{n\times k}$, $B \in \R^{k\times m}$\\
Rows of first matrix ``line up" with columns of the second matrix.\\
\Fonteight
$\left[\begin{matrix}
\rowcolor{red!20}    a_{1,1} & \hdots & a_{1,k} \\ 
\rowcolor{blue!20}   a_{2,1} & \hdots & a_{2,k} \\ 
                      \vdots & \hdots & \vdots \\
\rowcolor{green!20}  a_{n-1,1} & \hdots & a_{n-1,k} \\ 
\rowcolor{yellow!20} a_{n,1} & \hdots & a_{n,k} \\ 
\end{matrix}\right]$
$\left[\begin{array}{c>{\columncolor{gray!20}}cccc}
b_{1,1} & b_{1,2} &\hdots & b_{1,m-1} & b_{1,m} \\ 
\vdots & \vdots & \hdots & \vdots & \vdots \\
b_{k,1} &  b_{k,2} &\hdots & b_{k,m-1} & b_{k,m} 
\end{array}\right]$
$=
\left[\begin{array}{ccccc}
\hdots & {\cellcolor{red!20}}    \sum_{i=0}^k a_{1,i}b_{i,2} & \hdots & & \\
\hdots & {\cellcolor{blue!20}}   \sum_{i=0}^k a_{2,i}b_{i,2} & \hdots & & \\
\hdots &                        \vdots & \hdots & & \\
\hdots & {\cellcolor{green!20}}  \sum_{i=0}^k a_{n-1,i}b_{i,2} & \hdots& & \\
\hdots & {\cellcolor{yellow!20}} \sum_{i=0}^k a_{n,i}b_{i,2}& \hdots& & \\
\end{array}\right]
$
\end{frame}

\begin{frame}
\frametitle{Matrix Multiplication Mechanics: Inner Products}
(We haven't defined inner product yet)\\
Let $A\in \R^{n\times k}$, $B \in \R^{k\times m}$\\
Rows of first matrix ``line up" with columns of the second matrix.\\
\Fonteight
$\left[\begin{matrix}
\rowcolor{red!20}    a_{1,1} & \hdots & a_{1,k} \\ 
\rowcolor{blue!20}   a_{2,1} & \hdots & a_{2,k} \\ 
                      \vdots & \hdots & \vdots \\
\rowcolor{green!20}  a_{n-1,1} & \hdots & a_{n-1,k} \\ 
\rowcolor{yellow!20} a_{n,1} & \hdots & a_{n,k} \\ 
\end{matrix}\right]$
$\left[\begin{array}{cccc>{\columncolor{gray!20}}c}
b_{1,1} & b_{1,2} &\hdots & b_{1,m-1} & b_{1,m} \\ 
\vdots & \vdots & \hdots & \vdots & \vdots \\
b_{k,1} &  b_{k,2} &\hdots & b_{k,m-1} & b_{k,m} 
\end{array}\right]$
$=
\left[\begin{array}{ccccc}
\hdots & \hdots & {\cellcolor{red!20}}    \sum_{i=0}^k a_{1,i}b_{i,m}  \\
\hdots & \hdots & {\cellcolor{blue!20}}   \sum_{i=0}^k a_{2,i}b_{i,m}  \\
\hdots & \hdots &                        \vdots & \\
\hdots & \hdots & {\cellcolor{green!20}}  \sum_{i=0}^k a_{n-1,i}b_{i,m}  \\
\hdots & \hdots &{\cellcolor{yellow!20}} \sum_{i=0}^k a_{n,i}b_{i,m} \\
\end{array}\right]
$
\end{frame}
\begin{frame}
\frametitle{Matrix Multiplication Mechanics: Inner Products}
Inner Product method is the exact definition of matrix multiplication.\\
Each entry of the resultant matrix is an inner product of a row of the first matrix and a column of the second matrix.
\end{frame}


\begin{frame}
\frametitle{More M.M.M: Linear Combination of Columns}
Each column of the result is a linear combination of the columns of $A$.\\
$
\left[\begin{array}{>{\columncolor{red!20}}c>{\columncolor{blue!20}}cc>{\columncolor{green!20}}c >{\columncolor{yellow!20}}c}
\vline & \vline & \hdots & \vline  & \vline \\ 
\mathbf{a_1}    & \mathbf{a_2}    & \hdots & \mathbf{a_{k-1}}   &  \mathbf{a_k} \\
\vline & \vline & \hdots & \vline  & \vline \\ 
\end{array}\right]
$
$
\left[\begin{matrix}
{\cellcolor{red!20}}    b_{1,1}       & \hdots &   b_{1,m}  \\ 
{\cellcolor{blue!20}}   b_{2,1}       & \hdots &   b_{2,m}  \\ 
                          \vdots   & \vdots &   \vdots \\
{\cellcolor{green!20}} b_{k-1,1}  & \hdots &   b_{k-1,m}  \\ 
{\cellcolor{yellow!20}} b_{k,1}  & \hdots &   b_{k,m}
\end{matrix}\right]
$
$=
\left[\begin{array}{>{\columncolor{gray!20}}ccc}
\vline  & \hdots   & \vline \\ 
\sum_{i=1}^k \mathbf{a_i} b_{i,1}   & \hdots &  \sum_{i=1}^k \mathbf{a_i} b_{k,m} \\
\vline  & \hdots  & \vline \\ 
\end{array}\right]
$

\end{frame}

\begin{frame}
\frametitle{More M.M.M: Linear Combination of Columns}
Each column of the result is a linear combination of the columns of $A$.\\
%{>{\columncolor{red!20}}c>{\columncolor{blue!20}}c>{\columncolor{green!20}}cc >{\columncolor{yellow!20}}c}
$
\left[\begin{array}{>{\columncolor{red!20}}c>{\columncolor{blue!20}}cc>{\columncolor{green!20}}c >{\columncolor{yellow!20}}c}
\vline & \vline & \hdots & \vline  & \vline \\ 
\mathbf{a_1}    & \mathbf{a_2}    & \hdots & \mathbf{a_{k-1}}   &  \mathbf{a_k} \\
\vline & \vline & \hdots & \vline  & \vline \\ 
\end{array}\right]
$
$
\left[\begin{matrix}
b_{1,1}    & \hdots &   {\cellcolor{red!20}}  b_{1,m}  \\ 
b_{2,1}    & \hdots &   {\cellcolor{blue!20}}   b_{2,m}  \\ 
  \vdots   & \vdots &   \vdots \\
b_{k-1,1}  & \hdots &   {\cellcolor{green!20}} b_{k-1,m}  \\ 
b_{k,1}    & \hdots &   {\cellcolor{yellow!20}}b_{k,m}
\end{matrix}\right]
$
$=
\left[\begin{array}{cc>{\columncolor{gray!20}}c}
\vline  & \hdots   & \vline \\ 
\sum_{i=1}^k \mathbf{a_i} b_{i,1}   & \hdots &  \sum_{i=1}^k \mathbf{a_i} b_{k,m} \\
\vline  & \hdots  & \vline \\ 
\end{array}\right]
$

\end{frame}

\begin{frame}
\frametitle{More M.M.M: Linear Combination of Columns}
One dimensional case:\\
$
\left[\begin{array}{>{\columncolor{red!20}}c>{\columncolor{blue!20}}cc>{\columncolor{green!20}}c >{\columncolor{yellow!20}}c}
\vline & \vline & \hdots & \vline  & \vline \\ 
\mathbf{a_1}    & \mathbf{a_2}    & \hdots & \mathbf{a_{k-1}}   &  \mathbf{a_k} \\
\vline & \vline & \hdots & \vline  & \vline \\ 
\end{array}\right]
$
$
\left[\begin{matrix}
{\cellcolor{red!20}}    b_{1,1}  \\ 
{\cellcolor{blue!20}}   b_{2,1}  \\
                          \vdots \\
{\cellcolor{green!20}} b_{k-1,1} \\
{\cellcolor{yellow!20}} b_{k,1}  \\
\end{matrix}\right]
$
$=
\left[\begin{array}{>{\columncolor{gray!20}}c}
\vline \\
\sum_{i=1}^k \mathbf{a_i} b_{i,1} \\
\vline   \\
\end{array}\right]
$\\

Result is in the span of columns of A!
(Keep this in mind for later).
\end{frame}



\end{document}
